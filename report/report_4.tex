\documentclass{article}

\usepackage{graphicx}
\usepackage{float}
\usepackage{caption}

\usepackage{comment}
\usepackage[arrowdel]{physics}
\usepackage{amsfonts}
\usepackage{amsthm}
\DeclareMathOperator*{\argmax}{arg\,max}
\DeclareMathOperator*{\est}{Est}
\def\ind{\hspace{\parindent}}
\title{Brief Report}
\author{Amirreza Negari, Amirhossein Estiri, Sajad Kahani}
\date{Augest 2019}
\begin{document}

\maketitle

\section{Optimization Problem}
Let ${\lambda_1\dots\lambda_m}$ be set of free degrees of Hamiltonian, $\ket{\phi_0}$ the ground state and $\ket{\psi}$ the target state.

Formally, the problem can be formulated by this equation
\[ \lambda_1^*\dots\lambda_m^*, t^* = \argmax_{\lambda_1..\lambda_m, t} \mathcal{F}(e^{iH(\lambda_1\dots\lambda_m)t} \ket{\phi_0}, \ket{\psi}) \]

Where $\mathcal{F}(., .)$ is fidelity.

Specifically, we are focusing on a Hisenberg model, therefore

\[ H(J_1\dots J_{N-1}, B_1\dots B_N) = \sum_{i=1}^{N-1} J_i (X_i X_{i+1} + Y_i Y_{i+1}) + \sum_{i=1}^N B_i (1 - Z_i) \]

In this case, by multiplying each degrees by $\alpha$, and dividing $t$ by $\alpha$ result remains the same, therefore we can substitute $t$ with $t_0$.

\begin{equation} 
\label{eq:opt}
J_1^*\dots J_{N-1}^*, B_1^*\dots B_N^* = \argmax_{J_1\dots J_{N-1}, B_1\dots B_N} \mathcal{F}(e^{iHt_0} \ket{\phi_0}, \ket{\psi})
\end{equation}


\section{Hardness of Optimzation}

\newtheorem{theorem}{Theorem}

We can rewrite the whole problem as below
\[ \va*{\lambda}^* = \argmax_{\va*\lambda \in \Lambda} f(\va*{\lambda}) \]

Then we applied a bunch of assumptions to help sovling optimization problem.

Note that all of these statements are valid for arbitrary ground state and target state.

\begin{itemize}
 \item the places which $\grad_\Lambda f(\va*\lambda) = 0$ is not only limited to $\va*\lambda^*$.

 \item generated states by a hamiltonion through different parameters doesn't make a convex set.
 
 \item $f$ is not a convex/concave function of $\va\lambda$.
 
 \item $|\grad_\Lambda f|$ is not a function of $f$
\end{itemize}

the only guess that remains neither accepted or rejected is this theorem.

\begin{theorem}
\label{theor:monotonic}
  for all $\va*{\lambda}$ there exists a continious path (that can be parameterized as $\va{P}(u)$) that $\va{P}(0) = \va*{\lambda}$
and $\va{P}(1) = \va*{\lambda}^*$ and $f(\va{P}(u))$ is monotonically increasing. not proved yet!
\end{theorem}

\section{Numerical Methods}
A wide range of optimization algorithms applied or studied in order to help solving this problem. Behavior of Steepest-Descent (first order) optimizers analyzed and the result was that a simple Gradient-Descent will be effective in case of hyperparameter tuning.

Also BFGS as an applicable higher-order optimizer and NEWUOA, a derivative-free method was studied but have not been applied yet.

following results are from tuned version of Gradient-Descent.
\begin{table}[H]
\[ \begin{array}{c|c|c|c|c||c}
\dim \mathcal{H} & |S| & d & \bar{f} & f_{\min} & \text{MGF} \\ \hline 
2 & 20069 & 0.995 & 0.999997 & 0.99990 & 0.993 \\
3 & 101650 & 0.980 & 0.9998 & 0.9990 & 0.970 \\

4 & 162820 & 0.955 & 0.990 & 0.950 & 0.819 \\
\end{array} \]
\caption{one-sector subspace}
\end{table}

Improvement in result is obvious in comparison to our previous results. In addition to that, in state transfer we have reached slight improvement in comparison to baseline results that reported by [Bose 2003].

% TODO biblography

\section{Literature}
\begin{enumerate}
\item Kalinin, K. P., \& Berloff, N. G. (2018). Global optimization of spin Hamiltonians with gain-dissipative systems. Scientific reports, 8(1), 17791.
\item Quantum Technology and Optimization Problems: First International Workshop, QTOP 2019, Munich, Germany, March 18, 2019, Proceedings
\end{enumerate}
\end{document}
