\documentclass{article}
\usepackage{physics}
\def\ind{\hspace{\parindent}}
\begin{document}

\section{Similar Works}
\subsection{Deep Reinforcement Learning for Quantum Gate Control}
\subsubsection{Summary}
The paper states that approximation of unitary evolution in a quantum linear control (due to Pontryagin's principle) will be equivalent to choice of a sequence of unitary matrices from a limited set.

To find the proper sequence, an RL problem can be defined:
\begin{itemize}
  \item time is discretized.
  \item current state is accumulative evolution from the beginning. (continuous state space)
  \item action is the current evolution which is a function of control parameter. (limited discrete actions)
  \item transitions are totally certain (but will be modeled uncertainly).
  \item reward (at the last turn) is related to fidelity.
\end{itemize}

\subsubsection{Notes}
\ind p.2, \emph{``$d=16$''}: in a 4-d space, a hypercube has $2^4$ corners. it can be proved in a linear system corners are solution not planes (which has degrees of freedom).

p.4, \emph{``$\text{argmax}_a$''}: typo. it must be $\max_a$

p.5, \emph{``state transfer problem''}: lack of reasoning why RL can escape the trap, or reasoning on meaningfulness of experiments numbers that shows performance outreach (e.q. escaping the trap) 


\subsection{Other Works}
GRAPE, Genetics Algorithm

\section{Problem}
Suppose you have a number of N qubits, which are set to the state $\ket{10000}$ (I assume for definiteness that $N=5$).  How you can use machine learning to transform this state to a demanded state given by

\[ \ket{\psi} = a_1\ket{10000} + a_2\ket{01000} +
  a_3\ket{00100} + a_4\ket{00010} + a_5\ket{00001} \]

Here we say that the evolution is restricted to the one-particle sector (since there is only one single 1 in a background of zeros). 

You can then try to generalize this problem to arbitrary N and to the case where the number of 1's in each state is k (to the k-particle sector). 
\subsection{Methods}
\begin{itemize}
  \item Solovey-Kitaev Algorithm (needs modification)
  \item A* Search (proper hueristics)
  \item Text-like Generative Model (needs investigation)
\end{itemize}
\end{document}